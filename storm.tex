\documentclass[11pt]{article}
%Gummi|065|=)
\title{\textbf{Welcome to Gummi 0.6.6}}
\author{Gonzalo Manuel Garcia\\
		Mar\'ia Luc\'ia Pappaterra\\
		Dion Timmermann}
\date{}
\begin{document}

\maketitle

\section{Introducci\'on}

dejamos la introduccion para el final

 en You are now using Gummi 0.6.6. Many new exciting features have been added to the 0.6 series. The document editor is now a tabbed instance, allowing multiple documents to be worked on simultaneously. Using the new projects menu, you can group files together for easy access. 

Support for two high-level {\LaTeX} building systems, \emph{rubber}\footnote{https://launchpad.net/rubber/} \& \emph{latexmk}\footnote{http://www.phys.psu.edu/{\textasciitilde}collins/software/latexmk-jcc/} has been added as well. Your preferred typesetter can be configured through the Compilation tab in the Preferences menu. Typesetters that are not installed on your system will not be selectable. 

Added for your viewing convenience is a continuous preview mode for the PDF. This mode is enabled by default, but can also be disabled through the \emph{(View $\rightarrow$ Page layout in preview)} menu. Complementary to this feature is SyncTeX integration, which allows you to synchronize the position in your editor with the PDF preview. 

\section{Comienzos de Storm (contexto de creacion de la herramienta)}
Storm fue desarrollado en la Universidad de RWTH Aachen, Alemania por un grupo de estudiantes de doctorado y su profesor Joost-Pieter Katoen en el a\~no 2012 cuando en la \'ultima decada se estuv\'o desarrollando y madurando las herramientas probabil\'isticas de chequeo de modelos, Storm se desarroll\'o para tener una plataforma, facil de usar, y para experimentar con nuevos algoritmos de verificacion, modelos probabilisticamente ricos, algoritmos mejorados y diferentes formalismos de modelado.

Actualmente se usa academicamente , tanto los mismos desarrolladores como academias independientes o academias bajo contrato por la industria, para analizar sistemas aleatorios o con fenomenos probabilisticos, como algoritmos distribuidos (cuando la aleatorizacion corrompe la simetria entre procesos), seguridad (aleatorizacion de la generacion de claves), sistemas biologicos (especies o moleculas interactuando alatoriamente dependiendo de su concentracion), o sistemas enbebidos (involucrando las fallas de hardware). 



We hope you will enjoy using this release as much as we enjoyed creating it. If you have comments, suggestions or wish to report an issue you are experiencing - contact us at: \emph{https://github.com/alexandervdm/gummi}.

\section{Casos de estudio}


\section{Modelo basado en el Analisis de seguridad para sistema de guia para vehiculos}


\end{document}
